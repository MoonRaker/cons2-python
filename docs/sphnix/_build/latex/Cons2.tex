%% Generated by Sphinx.
\def\sphinxdocclass{report}
\documentclass[letterpaper,10pt,english]{sphinxmanual}
\ifdefined\pdfpxdimen
   \let\sphinxpxdimen\pdfpxdimen\else\newdimen\sphinxpxdimen
\fi \sphinxpxdimen=49336sp\relax

\usepackage[margin=1in,marginparwidth=0.5in]{geometry}
\usepackage[utf8]{inputenc}
\ifdefined\DeclareUnicodeCharacter
  \DeclareUnicodeCharacter{00A0}{\nobreakspace}
\fi
\usepackage{cmap}
\usepackage[T1]{fontenc}
\usepackage{amsmath,amssymb,amstext}
\usepackage{babel}
\usepackage{times}
\usepackage[Bjarne]{fncychap}
\usepackage{longtable}
\usepackage{sphinx}

\usepackage{multirow}
\usepackage{eqparbox}

% Include hyperref last.
\usepackage{hyperref}
% Fix anchor placement for figures with captions.
\usepackage{hypcap}% it must be loaded after hyperref.
% Set up styles of URL: it should be placed after hyperref.
\urlstyle{same}

\addto\captionsenglish{\renewcommand{\figurename}{Fig.\@ }}
\addto\captionsenglish{\renewcommand{\tablename}{Table }}
\addto\captionsenglish{\renewcommand{\literalblockname}{Listing }}

\addto\extrasenglish{\def\pageautorefname{page}}

\setcounter{tocdepth}{3}
\setcounter{secnumdepth}{3}


\title{Cons2 Documentation}
\date{Mar 22, 2017}
\release{3.0}
\author{Derek Groenendyk}
\newcommand{\sphinxlogo}{}
\renewcommand{\releasename}{Release}
\makeindex

\begin{document}

\maketitle
\sphinxtableofcontents
\phantomsection\label{\detokenize{index::doc}}


MetPy is a collection of tools in Python for reading, visualizing, and
performing calculations with weather data.

MetPy is still in an early stage of development, and as such
\sphinxstylestrong{no APIs are considered stable.} While we won't break things
just for fun, many things may still change as we work through
design issues.

We support Python 2.7 as well as Python \textgreater{}= 3.3.


\chapter{Documentation}
\label{\detokenize{index:cons2}}\label{\detokenize{index:documentation}}

\section{Overview}
\label{\detokenize{overview:overview}}\label{\detokenize{overview::doc}}\label{\detokenize{overview:id1}}

\subsection{Introduction}
\label{\detokenize{overview:introduction}}
The core mission of pvlib-python is to provide open, reliable,
interoperable, and benchmark implementations of PV system models.

There are at least as many opinions about how to model PV systems as
there are modelers of PV systems, so
pvlib-python provides several modeling paradigms.


\subsection{Modeling paradigms}
\label{\detokenize{overview:id2}}\label{\detokenize{overview:modeling-paradigms}}
The backbone of pvlib-python
is well-tested procedural code that implements PV system models.
pvlib-python also provides a collection of classes for users
that prefer object-oriented programming.


\subsubsection{Procedural}
\label{\detokenize{overview:procedural}}
The straightforward procedural code can be used for all modeling
steps in pvlib-python.


\subsection{Getting support}
\label{\detokenize{overview:getting-support}}
The best way to get support is to make an issue on our
\href{https://github.com/pvlib/pvlib-python/issues}{GitHub issues page} .


\subsection{How do I contribute?}
\label{\detokenize{overview:how-do-i-contribute}}
We're so glad you asked! Please see our
\href{https://github.com/pvlib/pvlib-python/wiki/Contributing-to-pvlib-python}{wiki}
for information and instructions on how to contribute.
We really appreciate it!


\subsection{Credits}
\label{\detokenize{overview:credits}}
USGS Arizona Water Science Center Water Use Program for support and testing.


\section{Installation}
\label{\detokenize{installation:installation}}\label{\detokenize{installation::doc}}\label{\detokenize{installation:id1}}

\section{Input Files}
\label{\detokenize{input_files:input-files}}\label{\detokenize{input_files::doc}}\label{\detokenize{input_files:id1}}

\section{Output Files}
\label{\detokenize{output_files::doc}}\label{\detokenize{output_files:output-files}}\label{\detokenize{output_files:id1}}

\section{Running CONS2}
\label{\detokenize{run:running-cons2}}\label{\detokenize{run::doc}}\label{\detokenize{run:run}}

\section{Contribution}
\label{\detokenize{contribution::doc}}\label{\detokenize{contribution:contribution}}\label{\detokenize{contribution:id1}}

\section{API reference}
\label{\detokenize{api/index:api-reference}}\label{\detokenize{api/index::doc}}\label{\detokenize{api/index:api-index}}

\subsection{Classes}
\label{\detokenize{api/api::doc}}\label{\detokenize{api/api:api}}\label{\detokenize{api/api:classes}}
pvlib-python provides a collection of classes for users that prefer
object-oriented programming. These classes can help users keep track of
data in a more organized way, and can help to simplify the modeling
process. The classes do not add any functionality beyond the procedural
code. Most of the object methods are simple wrappers around the
corresponding procedural code.

\begin{longtable}{p{0.5\linewidth}p{0.5\linewidth}}
\hline
\endfirsthead

\multicolumn{2}{c}%
{{\tablecontinued{\tablename\ \thetable{} -- continued from previous page}}} \\
\hline
\endhead

\hline \multicolumn{2}{|r|}{{\tablecontinued{Continued on next page}}} \\ \hline
\endfoot

\endlastfoot


{\hyperref[\detokenize{api/generated/cons2.cu.CONSUMPTIVE_USE:cons2.cu.CONSUMPTIVE_USE}]{\sphinxcrossref{\sphinxcode{cu.CONSUMPTIVE\_USE}}}}(sp, cp)
&
docstring for CONSUMPTIVE\_USE
\\
\hline
{\hyperref[\detokenize{api/generated/cons2.crop.CROP:cons2.crop.CROP}]{\sphinxcrossref{\sphinxcode{crop.CROP}}}}(shrtname, longname, crop\_type, ...)
&
docstring for CROP
\\
\hline
{\hyperref[\detokenize{api/generated/cons2.site.SITE:cons2.site.SITE}]{\sphinxcrossref{\sphinxcode{site.SITE}}}}(sws)
&
docstring for SITE
\\
\hline
{\hyperref[\detokenize{api/generated/cons2.weather.WEATHER:cons2.weather.WEATHER}]{\sphinxcrossref{\sphinxcode{weather.WEATHER}}}}(fname, wsname{[}, units{]})
&
docstring for WEATHER
\\
\hline
{\hyperref[\detokenize{api/generated/cons2.excel.Excel:cons2.excel.Excel}]{\sphinxcrossref{\sphinxcode{excel.Excel}}}}(filename{[}, vis, create{]})
&

\\
\hline\end{longtable}



\subsubsection{cons2.cu.CONSUMPTIVE\_USE}
\label{\detokenize{api/generated/cons2.cu.CONSUMPTIVE_USE:cons2-cu-consumptive-use}}\label{\detokenize{api/generated/cons2.cu.CONSUMPTIVE_USE::doc}}\index{CONSUMPTIVE\_USE (class in cons2.cu)}

\begin{fulllineitems}
\phantomsection\label{\detokenize{api/generated/cons2.cu.CONSUMPTIVE_USE:cons2.cu.CONSUMPTIVE_USE}}\pysiglinewithargsret{\sphinxstrong{class }\sphinxcode{cons2.cu.}\sphinxbfcode{CONSUMPTIVE\_USE}}{\emph{sp}, \emph{cp}}{}
docstring for CONSUMPTIVE\_USE
\index{\_\_init\_\_() (cons2.cu.CONSUMPTIVE\_USE method)}

\begin{fulllineitems}
\phantomsection\label{\detokenize{api/generated/cons2.cu.CONSUMPTIVE_USE:cons2.cu.CONSUMPTIVE_USE.__init__}}\pysiglinewithargsret{\sphinxbfcode{\_\_init\_\_}}{\emph{sp}, \emph{cp}}{}
\end{fulllineitems}

\paragraph{Methods}

\begin{longtable}{p{0.5\linewidth}p{0.5\linewidth}}
\hline
\endfirsthead

\multicolumn{2}{c}%
{{\tablecontinued{\tablename\ \thetable{} -- continued from previous page}}} \\
\hline
\endhead

\hline \multicolumn{2}{|r|}{{\tablecontinued{Continued on next page}}} \\ \hline
\endfoot

\endlastfoot


{\hyperref[\detokenize{api/generated/cons2.cu.CONSUMPTIVE_USE:cons2.cu.CONSUMPTIVE_USE.__init__}]{\sphinxcrossref{\sphinxcode{\_\_init\_\_}}}}(sp, cp)
&

\\
\hline
\sphinxcode{calc\_adj}(cuirr, pccrop, precip, nbegmo, day3)
&

\\
\hline
\sphinxcode{calc\_cu}()
&

\\
\hline
\sphinxcode{calc\_dates}()
&

\\
\hline
\sphinxcode{calc\_effprecip}(precip)
&

\\
\hline
\sphinxcode{calc\_fao}(yr{[}, repeat{]})
&

\\
\hline
\sphinxcode{calc\_faokc}(frac, lfrac, kc)
&

\\
\hline
\sphinxcode{calc\_kc}()
&

\\
\hline
\sphinxcode{calc\_midpts}()
&
Find midpoints of seasons
\\
\hline
\sphinxcode{calc\_pclite}(lat)
&

\\
\hline
\sphinxcode{calc\_temp}()
&

\\
\hline
\sphinxcode{clndr}(doy)
&
Convert day of year in month and day.
\\
\hline
\sphinxcode{fall}(atemp, mean)
&
Find end of growing season day of year.
\\
\hline
\sphinxcode{fao\_cu}(p, tavg)
&

\\
\hline
\sphinxcode{fiveyr\_avg}(yr, fivc, fivcr, pcu, pcuirr)
&

\\
\hline
\sphinxcode{get\_dates}(temps)
&
Find the start and end of the
\\
\hline
\sphinxcode{interp\_kc}(mid, temp, day)
&

\\
\hline
\sphinxcode{jln}(m, d)
&

\\
\hline
\sphinxcode{kc\_ann}()
&

\\
\hline
\sphinxcode{kc\_per}()
&

\\
\hline
\sphinxcode{midday}(nmo, midpt, day2, k, pclite, middle)
&

\\
\hline
\sphinxcode{midtemp}(nmo, midpt, k, temp, middle)
&
Calculates mean monthly temperature
\\
\hline
\sphinxcode{mmtemp}(nmo, midpt, day2, num)
&
Caclulates Spring part month mean temperature
\\
\hline
\sphinxcode{set\_dates}()
&

\\
\hline
\sphinxcode{spring}(atemp, mean)
&
Find beginning of growing season day of year.
\\
\hline\end{longtable}


\end{fulllineitems}



\subsubsection{cons2.crop.CROP}
\label{\detokenize{api/generated/cons2.crop.CROP::doc}}\label{\detokenize{api/generated/cons2.crop.CROP:cons2-crop-crop}}\index{CROP (class in cons2.crop)}

\begin{fulllineitems}
\phantomsection\label{\detokenize{api/generated/cons2.crop.CROP:cons2.crop.CROP}}\pysiglinewithargsret{\sphinxstrong{class }\sphinxcode{cons2.crop.}\sphinxbfcode{CROP}}{\emph{shrtname}, \emph{longname}, \emph{crop\_type}, \emph{mmnum}, \emph{directory}, \emph{sp}}{}
docstring for CROP
\index{\_\_init\_\_() (cons2.crop.CROP method)}

\begin{fulllineitems}
\phantomsection\label{\detokenize{api/generated/cons2.crop.CROP:cons2.crop.CROP.__init__}}\pysiglinewithargsret{\sphinxbfcode{\_\_init\_\_}}{\emph{shrtname}, \emph{longname}, \emph{crop\_type}, \emph{mmnum}, \emph{directory}, \emph{sp}}{}
\end{fulllineitems}

\paragraph{Methods}

\begin{longtable}{p{0.5\linewidth}p{0.5\linewidth}}
\hline
\endfirsthead

\multicolumn{2}{c}%
{{\tablecontinued{\tablename\ \thetable{} -- continued from previous page}}} \\
\hline
\endhead

\hline \multicolumn{2}{|r|}{{\tablecontinued{Continued on next page}}} \\ \hline
\endfoot

\endlastfoot


{\hyperref[\detokenize{api/generated/cons2.crop.CROP:cons2.crop.CROP.__init__}]{\sphinxcrossref{\sphinxcode{\_\_init\_\_}}}}(shrtname, longname, crop\_type, ...)
&

\\
\hline
\sphinxcode{get\_ckc}()
&
Reads in crop coefficients.
\\
\hline
\sphinxcode{get\_nckc}()
&
Reads in crop coefficients.
\\
\hline
\sphinxcode{read\_cropdev}()
&

\\
\hline\end{longtable}


\end{fulllineitems}



\subsubsection{cons2.site.SITE}
\label{\detokenize{api/generated/cons2.site.SITE::doc}}\label{\detokenize{api/generated/cons2.site.SITE:cons2-site-site}}\index{SITE (class in cons2.site)}

\begin{fulllineitems}
\phantomsection\label{\detokenize{api/generated/cons2.site.SITE:cons2.site.SITE}}\pysiglinewithargsret{\sphinxstrong{class }\sphinxcode{cons2.site.}\sphinxbfcode{SITE}}{\emph{sws}}{}
docstring for SITE
\index{\_\_init\_\_() (cons2.site.SITE method)}

\begin{fulllineitems}
\phantomsection\label{\detokenize{api/generated/cons2.site.SITE:cons2.site.SITE.__init__}}\pysiglinewithargsret{\sphinxbfcode{\_\_init\_\_}}{\emph{sws}}{}
\end{fulllineitems}

\paragraph{Methods}

\begin{longtable}{p{0.5\linewidth}p{0.5\linewidth}}
\hline
\endfirsthead

\multicolumn{2}{c}%
{{\tablecontinued{\tablename\ \thetable{} -- continued from previous page}}} \\
\hline
\endhead

\hline \multicolumn{2}{|r|}{{\tablecontinued{Continued on next page}}} \\ \hline
\endfoot

\endlastfoot


{\hyperref[\detokenize{api/generated/cons2.site.SITE:cons2.site.SITE.__init__}]{\sphinxcrossref{\sphinxcode{\_\_init\_\_}}}}(sws)
&

\\
\hline
\sphinxcode{import\_crops}()
&

\\
\hline
\sphinxcode{read\_cropfile}()
&

\\
\hline
\sphinxcode{read\_sitefile}()
&

\\
\hline\end{longtable}


\end{fulllineitems}



\subsubsection{cons2.weather.WEATHER}
\label{\detokenize{api/generated/cons2.weather.WEATHER::doc}}\label{\detokenize{api/generated/cons2.weather.WEATHER:cons2-weather-weather}}\index{WEATHER (class in cons2.weather)}

\begin{fulllineitems}
\phantomsection\label{\detokenize{api/generated/cons2.weather.WEATHER:cons2.weather.WEATHER}}\pysiglinewithargsret{\sphinxstrong{class }\sphinxcode{cons2.weather.}\sphinxbfcode{WEATHER}}{\emph{fname}, \emph{wsname}, \emph{units='metric'}}{}
docstring for WEATHER
\index{\_\_init\_\_() (cons2.weather.WEATHER method)}

\begin{fulllineitems}
\phantomsection\label{\detokenize{api/generated/cons2.weather.WEATHER:cons2.weather.WEATHER.__init__}}\pysiglinewithargsret{\sphinxbfcode{\_\_init\_\_}}{\emph{fname}, \emph{wsname}, \emph{units='metric'}}{}
\end{fulllineitems}

\paragraph{Methods}

\begin{longtable}{p{0.5\linewidth}p{0.5\linewidth}}
\hline
\endfirsthead

\multicolumn{2}{c}%
{{\tablecontinued{\tablename\ \thetable{} -- continued from previous page}}} \\
\hline
\endhead

\hline \multicolumn{2}{|r|}{{\tablecontinued{Continued on next page}}} \\ \hline
\endfoot

\endlastfoot


{\hyperref[\detokenize{api/generated/cons2.weather.WEATHER:cons2.weather.WEATHER.__init__}]{\sphinxcrossref{\sphinxcode{\_\_init\_\_}}}}(fname, wsname{[}, units{]})
&

\\
\hline
\sphinxcode{mnmnthly}(df)
&

\\
\hline
\sphinxcode{read\_data}()
&

\\
\hline\end{longtable}


\end{fulllineitems}



\subsubsection{cons2.excel.Excel}
\label{\detokenize{api/generated/cons2.excel.Excel:cons2-excel-excel}}\label{\detokenize{api/generated/cons2.excel.Excel::doc}}\index{Excel (class in cons2.excel)}

\begin{fulllineitems}
\phantomsection\label{\detokenize{api/generated/cons2.excel.Excel:cons2.excel.Excel}}\pysiglinewithargsret{\sphinxstrong{class }\sphinxcode{cons2.excel.}\sphinxbfcode{Excel}}{\emph{filename}, \emph{vis=True}, \emph{create=False}}{}~\index{\_\_init\_\_() (cons2.excel.Excel method)}

\begin{fulllineitems}
\phantomsection\label{\detokenize{api/generated/cons2.excel.Excel:cons2.excel.Excel.__init__}}\pysiglinewithargsret{\sphinxbfcode{\_\_init\_\_}}{\emph{filename}, \emph{vis=True}, \emph{create=False}}{}
\end{fulllineitems}

\paragraph{Methods}

\begin{longtable}{p{0.5\linewidth}p{0.5\linewidth}}
\hline
\endfirsthead

\multicolumn{2}{c}%
{{\tablecontinued{\tablename\ \thetable{} -- continued from previous page}}} \\
\hline
\endhead

\hline \multicolumn{2}{|r|}{{\tablecontinued{Continued on next page}}} \\ \hline
\endfoot

\endlastfoot


{\hyperref[\detokenize{api/generated/cons2.excel.Excel:cons2.excel.Excel.__init__}]{\sphinxcrossref{\sphinxcode{\_\_init\_\_}}}}(filename{[}, vis, create{]})
&

\\
\hline
\sphinxcode{close\_workbook}({[}save\_state{]})
&

\\
\hline
\sphinxcode{createSheet}(workbook, name)
&

\\
\hline
\sphinxcode{create\_workbook}({[}state{]})
&

\\
\hline
\sphinxcode{open}()
&

\\
\hline
\sphinxcode{open\_workbook}({[}state{]})
&

\\
\hline
\sphinxcode{setVis}(boolean)
&

\\
\hline\end{longtable}


\end{fulllineitems}



\subsection{Main}
\label{\detokenize{api/api:main}}
\begin{longtable}{p{0.5\linewidth}p{0.5\linewidth}}
\hline
\endfirsthead

\multicolumn{2}{c}%
{{\tablecontinued{\tablename\ \thetable{} -- continued from previous page}}} \\
\hline
\endhead

\hline \multicolumn{2}{|r|}{{\tablecontinued{Continued on next page}}} \\ \hline
\endfoot

\endlastfoot


\sphinxcode{main}
&

\\
\hline\end{longtable}



\subsubsection{cons2.main}
\label{\detokenize{api/generated/cons2.main::doc}}\label{\detokenize{api/generated/cons2.main:cons2-main}}\label{\detokenize{api/generated/cons2.main:module-cons2.main}}\index{cons2.main (module)}\paragraph{Functions}

\begin{longtable}{p{0.5\linewidth}p{0.5\linewidth}}
\hline
\endfirsthead

\multicolumn{2}{c}%
{{\tablecontinued{\tablename\ \thetable{} -- continued from previous page}}} \\
\hline
\endhead

\hline \multicolumn{2}{|r|}{{\tablecontinued{Continued on next page}}} \\ \hline
\endfoot

\endlastfoot


\sphinxcode{import\_data}(vis)
&
Reads the site input file and initilizes a SITE object, including crop and weather information.
\\
\hline
\sphinxcode{main}()
&

\\
\hline
\sphinxcode{run}()
&
Runs the CONS2 program.
\\
\hline
\sphinxcode{write\_crp\_output}(directory, name, cname, ...)
&
Writes output data for each crop.
\\
\hline
\sphinxcode{write\_excel\_monthly}(wb, name, etmethod, ...)
&

\\
\hline
\sphinxcode{write\_excel\_yearly}(wb, name, etmethod, ...)
&

\\
\hline
\sphinxcode{write\_output}(directory, name, years, data)
&
Writes summary data for consumptive use, effective precipitation, and irrigrated consumptive for the whole site.
\\
\hline\end{longtable}

\paragraph{Classes}

\begin{longtable}{p{0.5\linewidth}p{0.5\linewidth}}
\hline
\endfirsthead

\multicolumn{2}{c}%
{{\tablecontinued{\tablename\ \thetable{} -- continued from previous page}}} \\
\hline
\endhead

\hline \multicolumn{2}{|r|}{{\tablecontinued{Continued on next page}}} \\ \hline
\endfoot

\endlastfoot


\sphinxcode{SITE}(sws)
&
docstring for SITE
\\
\hline
\sphinxcode{date}
&
date(year, month, day) --\textgreater{} date object
\\
\hline
\sphinxcode{dt}
&
alias of \sphinxcode{datetime}
\\
\hline\end{longtable}



\subsection{Read Infile}
\label{\detokenize{api/api:read-infile}}
Creating a ModelChain object.

\begin{longtable}{p{0.5\linewidth}p{0.5\linewidth}}
\hline
\endfirsthead

\multicolumn{2}{c}%
{{\tablecontinued{\tablename\ \thetable{} -- continued from previous page}}} \\
\hline
\endhead

\hline \multicolumn{2}{|r|}{{\tablecontinued{Continued on next page}}} \\ \hline
\endfoot

\endlastfoot


\sphinxcode{read\_infile}
&

\\
\hline\end{longtable}



\subsubsection{cons2.read\_infile}
\label{\detokenize{api/generated/cons2.read_infile:module-cons2.read_infile}}\label{\detokenize{api/generated/cons2.read_infile::doc}}\label{\detokenize{api/generated/cons2.read_infile:cons2-read-infile}}\index{cons2.read\_infile (module)}\paragraph{Functions}

\begin{longtable}{p{0.5\linewidth}p{0.5\linewidth}}
\hline
\endfirsthead

\multicolumn{2}{c}%
{{\tablecontinued{\tablename\ \thetable{} -- continued from previous page}}} \\
\hline
\endhead

\hline \multicolumn{2}{|r|}{{\tablecontinued{Continued on next page}}} \\ \hline
\endfoot

\endlastfoot


\sphinxcode{get\_data}(fname)
&

\\
\hline
\sphinxcode{main}()
&

\\
\hline\end{longtable}

\paragraph{Classes}

\begin{longtable}{p{0.5\linewidth}p{0.5\linewidth}}
\hline
\endfirsthead

\multicolumn{2}{c}%
{{\tablecontinued{\tablename\ \thetable{} -- continued from previous page}}} \\
\hline
\endhead

\hline \multicolumn{2}{|r|}{{\tablecontinued{Continued on next page}}} \\ \hline
\endfoot

\endlastfoot


\sphinxcode{OrderedDict}
&
Dictionary that remembers insertion order
\\
\hline
\sphinxcode{islice}
&
islice(iterable, stop) --\textgreater{} islice object
\\
\hline\end{longtable}

\begin{itemize}
\item {} 
\DUrole{xref,std,std-ref}{modindex}

\item {} 
\DUrole{xref,std,std-ref}{genindex}

\end{itemize}


\section{Glossary}
\label{\detokenize{variables_style_rules:glossary}}\label{\detokenize{variables_style_rules::doc}}\label{\detokenize{variables_style_rules:variables-style-rules}}
There is a convention on consistent variable names throughout the program:

\begin{longtable}{|l|l|}
\caption{List of used Variables and Parameters}\label{\detokenize{variables_style_rules:id1}}\\
\hline
\sphinxstylethead{\relax 
variable
\unskip}\relax &\sphinxstylethead{\relax 
description
\unskip}\relax \\
\hline\endfirsthead

\multicolumn{2}{c}%
{{\tablecontinued{\tablename\ \thetable{} -- continued from previous page}}} \\
\hline
\sphinxstylethead{\relax 
variable
\unskip}\relax &\sphinxstylethead{\relax 
description
\unskip}\relax \\
\hline\endhead

\hline \multicolumn{2}{|r|}{{\tablecontinued{Continued on next page}}} \\ \hline
\endfoot

\endlastfoot


tz
&
timezone
\\
\hline
latitude
&
latitude
\\
\hline
longitude
&
longitude
\\
\hline
dni
&
direct normal irradiance
\\
\hline
dni\_extra
&
direct normal irradiance at top of atmosphere (extraterrestrial)
\\
\hline
dhi
&
diffuse horizontal irradiance
\\
\hline
ghi
&
global horizontal irradiance
\\
\hline
aoi
&
angle of incidence
\\
\hline
aoi\_projection
&
cos(aoi)
\\
\hline
airmass
&
airmass
\\
\hline
airmass\_relative
&
relative airmass
\\
\hline
airmass\_absolute
&
absolute airmass
\\
\hline
poa\_ground\_diffuse
&
in plane ground reflected irradiation
\\
\hline
poa\_direct
&
direct/beam irradiation in plane
\\
\hline
poa\_diffuse
&
total diffuse irradiation in plane. sum of ground and sky diffuse.
\\
\hline
poa\_global
&
global irradiation in plane. sum of diffuse and beam projection.
\\
\hline
poa\_sky\_diffuse
&
diffuse irradiation in plane from scattered light in the atmosphere (without ground reflected irradiation)
\\
\hline
g\_poa\_effective
&
broadband plane of array effective irradiance.
\\
\hline
surface\_tilt
&
tilt angle of the surface
\\
\hline
surface\_azimuth
&
azimuth angle of the surface
\\
\hline
solar\_zenith
&
zenith angle of the sun in degrees
\\
\hline
apparent\_zenith
&
refraction-corrected solar zenith angle in degrees
\\
\hline
solar\_azimuth
&
azimuth angle of the sun in degrees East of North
\\
\hline
temp\_cell
&
temperature of the cell
\\
\hline
temp\_module
&
temperature of the module
\\
\hline
temp\_air
&
temperature of the air
\\
\hline
temp\_dew
&
dewpoint temperature
\\
\hline
relative\_humidity
&
relative humidity
\\
\hline
v\_mp, i\_mp, p\_mp
&
module voltage, current, power at the maximum power point
\\
\hline
v\_oc
&
open circuit module voltage
\\
\hline
i\_sc
&
short circuit module current
\\
\hline
i\_x, i\_xx
&
Sandia Array Performance Model IV curve parameters
\\
\hline
effective\_irradiance
&
effective irradiance
\\
\hline
photocurrent
&
photocurrent
\\
\hline
saturation\_current
&
diode saturation current
\\
\hline
resistance\_series
&
series resistance
\\
\hline
resistance\_shunt
&
shunt resistance
\\
\hline
transposition\_factor
&
the gain ratio of the radiation on inclined plane to global horizontal irradiation: \(\frac{poa\_global}{ghi}\)
\\
\hline
pdc0
&
nameplate DC rating
\\
\hline
pdc, dc
&
dc power
\\
\hline
gamma\_pdc
&
module temperature coefficient. Typically in units of 1/C.
\\
\hline
pac, ac
&
ac powe.
\\
\hline
eta\_inv
&
inverter efficiency
\\
\hline
eta\_inv\_ref
&
reference inverter efficiency
\\
\hline
eta\_inv\_nom
&
nominal inverter efficiency
\\
\hline\end{longtable}


\begin{sphinxadmonition}{note}{Note:}
These further references might not use the same terminology as \sphinxstyleemphasis{cons2}. But the physical process referred to is the same.
\end{sphinxadmonition}


\chapter{Contact Us}
\label{\detokenize{index:contact-us}}\begin{itemize}
\item {} 
For questions and discussion about MetPy, join Unidata's \href{https://www.unidata.ucar.edu/support/\#mailinglists}{python-users}
mailing list

\item {} 
The source code is available on \href{https://github.com/Unidata/MetPy}{GitHub}

\item {} 
Bug reports and feature requests should be directed to the
\href{https://github.com/Unidata/MetPy/issues}{GitHub issue tracker}

\item {} 
MetPy has a \href{https://gitter.im/Unidata/MetPy}{Gitter} chatroom for more ``live'' communication

\item {} 
MetPy can also be found on \href{https://twitter.com/MetPy}{Twitter}

\end{itemize}


\chapter{License}
\label{\detokenize{index:license}}\label{\detokenize{index:id1}}
MetPy is available under the terms of the open source \href{https://raw.githubusercontent.com/Unidata/MetPy/master/LICENSE}{BSD 3 Clause license}.



\renewcommand{\indexname}{Index}
\printindex
\end{document}